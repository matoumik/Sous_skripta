\documentclass[fykos]{fksgeneric}
\usepackage{amsmath}

\begin{document}

\newtheorem{definice}{Definice}
\newtheorem{veta}{Věta}

\rightheader{soustředění jaro 2018}
%\centering
\hspace{2cm}
\section{Kvantový mišmaš}
\hspace{2cm}
\subsection{Poděkování}
Při sepisování tohoto textu jsem vycházel z knihy Úvod do kvantové mechaniky od
\textbf{Prof. RNDr. Lubomíra Skály, DrSc.}, kterou doporučuju hlubším zálemcům o toto téma.
\subsection{Úvod}
Někdy kolem roku 1900 se zjistilo, že fyzika mikrosvěta se chová minimálně \uv{divně}.
Na základě několika experimentů se vybudovala v první polovině 20. století komplexní teorie, která je základem moderních fyzikálních teorií.

\subsection{Úvod do formalismu}
V klasické mechanice je systém popisován pomocí dvou veličin, polohy a hybnosti, pro každý hmotný bod soustavy. To znamená pro 3D prostoru se jedná o $6N$ proměnných, 
tedy $6N$ dimenziální fázový prostor, který popisuje jakykoliv stav systému.

Naproti tomu prostor stavů v kvantovce je obecně nekonečnědimenzionální. Je popisován vlnovou funkcí $\Psi(x,t): \mathbb{R}^3 \rightarrow \mathbb{C}$, která v daném čase popisuje stav systému. Máme na ní několik požadavků. Předně musí být konečná, spojitá a pro rozumné potenciály pak i $C^1$. Samotná vlnová funkce nemá žáný přímý fyzikální význam, ale veličina $\abs{\Psi}^2 = \Psi^\ast\Psi = P\d V$ jeasterisc hustota pravděpodobnosti výskytu částice. Pravděpdobnost výskytu částice v nějaké množině $\Omega$ získáme přeintegrováním této veličiny přes tuto množinu. Proto získáváme třetí požadavek na
funkci, protože částice někde být musí, čímž získáme třetí požadavek
\eq{\int_{\mathbb{R}^3} \Psi^\ast\Psi \d x= 1.}
Toto odpovídá na prostoru funkcí skalárnímu součínu, čímž vlastně chceme, aby norma vektoru byla jednotková.

Fyzikální veličiny jsou naproti tomu reprezentovány operátory, které působí na vlnovou funkci systému. Už z podstaty výsledku je zřejmé, že nedostaneme konkrétní hodnotu, ale
zase nějakou funkci. Ale můžeme změřit střední hodnotu pomocí
\eq{\left\langle \hat{A} \right\rangle = \int_{\mathbb{R}^3} \Psi^\ast \hat{A} \Psi \d x.}
Pravděpodobnost naměření určité hodnoty získáme pomocí rozvoje $\Psi$ do vlastních funkcí daného operátoru.


\subsection{Schrödingerova rovnice}
Časový vývoj systému je dán speciálním operátorem nazývaným Hamiltonián, který je kvantovou podobou operátoru z klasické mechaniky. Na rozdíl od klasické mechaniky je definován jednoduššeji $\hat{H} = \hat{T} + \hat{V}$, kde $\hat{T}$ je operátor kinetické energie, definovaný většinou pomoc
\eq{\hat{T}=\frac{\hat{p}^2}{2m}=-\frac{\hbar^2}{2m}\Delta,}
a $\hat{V}$ je potenciál, definovaný pro daný problém, jehož chování hledáme.
Tento vývoj je dán časovou Schrödingerovou rovnicí
\eq{i\hbar\pder{\Psi}{t}=\hat{H}\Psi:}
Pokud tbar Hamiltoniánu nezávisí na čase, můžeme hledat řešení ve tvaru součinu funkce, která nezávisí na čase a funkce, která nezávisí na prostorových proměnných. Tím pak získáme dvě nezávislé diferenciální rovnice, které řešíme zvlášť.
Nás především zajímá ta pro prostorovou závislost funkce, která se nazývá nečasovou (bezčasovou Schrödingerovou rovnicí. Její tvar je
\eq{\hat{H}\Psi=E\Psi.}

Všimněte si, že na pravé straně je konstanta krát původní funkce, což efektivně říká, že uvedená rovnice je jen hledáním vlastních funkcí a vlastních čísel operátoru. Pokud najdeme ty, pak vývoj systému je dán pomocí
\eq{\Psi(x,t)=\sum_n c_n \Psi_n \eu^{-\im\frac{E_n}{\hbar}t}}
Pro časově závislý Hamiltonián nám nezbývá než zkusit řešit časovou Schrodingerovu rovnici přímo. Pokud je systém ve vlastním stavu hamiltoniánu, je hustota pravděpodobnosti nezávislá na čase a systém je \uv{zamrzlý}, což je jev, který z klasické mechaniky neznáme.

\subsection{Operátory}
I na operátory máme určité požadavky. Abychom mohli použít bohaté teorie z oblasti lineární algebry, chceme aby operátory byly lineární. Dále chceme, aby operátory, které odpovídající měřitelným veličinám byly hermitovské, z čehož jednak vyplývá, že mají reálné spektrum, druhak nedegenerované vlastní funkce jsou navzájem ortogonální, což nám často usnadní výpočty.
Máme především operátory souřadnice,
\eq{\hat{x}\Psi = x\Psi,\; \hat{y}\Psi = y\Psi,\; \hat{z}\Psi = z\Psi}
Ty tedy jen danou souřadnicí přenásobí vlnovou funkci.

Naopak složky hybnosti jsou definovány naopak pomocí
\eq{\hat{p}_x=-\im\hbar\pder{}{x},\; \hat{p}_y=-\im\hbar\pder{}{y},\; \hat{p}_z=-\im\hbar\pder{}{z}}

Máme-li dva operátory, zavádíme jejich komutátor pomocí
\eq{\left[\hat{A},\hat{B}\right]=\hat{A}\hat{B}-\hat{B}\hat{A}.}
Pokud komutátor je nulový, tak operátory mají hezčí vlastnosti. Můžeme prohazovat ve výpčtech operátory, ale hlavně tyto dva operátory mají společný systém vlastních funkcí,
čehož se často využívá.
\end{document}